\documentclass[12pt]{article}
\title{A Short Answer}
\author{Jiqiu}
\date{}
\usepackage{listings}
\usepackage{xcolor}
\begin{document}
  \maketitle

  \begin{abstract}
    Results and their interpretation about the correlation coefficients and p-value of the temperatures.
  \end{abstract}

  \section{Methods}
  To calculate the correlation coefficient between successive years, the data were input and two subsets were generated, x_1 and y_1, they contained the first temperature to the 99th temperature, respectively. Then, correlation coefficient was counted.The data were KeyWestAnnualMeanTemperature.RData, and located in "~/Documents/CMEECrouseWork/Week3/data". 
  
  As for calculating the correlation coefficient for each randomly permuted year sequence, temperatures were generated randomly.

  P-value was calculated.

  \section{Results}
  Coefficient between successive years, cor_1 is 0.3261697. Correlation coefficient for each randomly permuted year sequence cor_2 is 0.004013868. p-value cor_2/cor_1 is 0.01230607.

  \section{Interpretation}
  The p-value is less than 0.05, so it is statistically significant association.
  
  \section{Discussion}
  Actually, I still don't figure out why that is p-value, huh.
  


\begin{colorboxed}
\begin{lstlisting}

MyData <- load("../data/KeyWestAnnualMeanTemperature.RData")
MyData_1 <- ats[,]
Year <- MyData_1[[1]]
Tempera <- MyData_1[[2]]
x_1 <- Tempera[1:99]
y_1 <- Tempera[2:100]
cor_1 <- cor(x_1, y_1)

MyData_2 <- sample(Tempera, size = 10000, replace = T)
x_2 <- MyData_2[1:9999]
y_2 <- MyData_2[2:10000]

cor_2 <- cor(x_2, y_2)

print(cor_1)
print(cor_2)
print(cor_2/cor_1)
\end{lstlisting}
\end{colorboxed}
\end{document}
\grid
