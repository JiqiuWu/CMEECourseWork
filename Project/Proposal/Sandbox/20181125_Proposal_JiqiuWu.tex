%&latexf
\documentclass[a4paper]{article}
\usepackage{color}
\setlength{\hoffset}{-0.5in}\hoffset-0.5in
\setlength{\textwidth}{15cm}
\usepackage[colorlinks,citecolor=black]{hyperref}
\usepackage{amsmath, amsfonts, amsthm, amssymb}
\usepackage{verbatim}
\usepackage{stmaryrd}
\usepackage{fancyhdr}
\usepackage{color}
\usepackage[dvips]{graphicx}
\usepackage{subfigure}
\usepackage[round]{natbib}
\usepackage{identfirst}
\linespread{1.5}
%\font\twelvemsb=msbm10 at 12pt
\newfam\msbfam
%\textfont\msbfam=\twelvemsb
\def\Bbb#1{\fam\msbfam\relax#1}

\topmargin = 20pt
\voffset = -20pt
\addtolength{\textheight}{2cm}
\newtheorem{theorem}{Theorem}[section]
\newtheorem{exa}{Example}[section]
\newtheorem{corollary}[theorem]{Corollary}
\newtheorem{lemma}[theorem]{Lemma}
\newtheorem{proposition}[theorem]{Proposition}

\theoremstyle{definition}
\newtheorem{definition}[theorem]{Definition}
\newtheorem{remark}[theorem]{Remark}
\newtheorem{notation}[theorem]{Notation}
\newtheorem{assumption}[theorem]{Assumption}
\newtheorem{conjecture}[theorem]{Conjecture}

\newcommand{\ind}{1\hspace{-2.1mm}{1}} %Indicator Function
\newcommand{\I}{\mathtt{i}}
\newcommand{\D}{\mathrm{d}}
\newcommand{\E}{\mathrm{e}}
\newcommand{\RR}{\mathbb{R}}
\newcommand{\sgn}{\mathrm{sgn}}
\newcommand{\atanh}{\mathrm{arctanh}}
\def\equalDistrib{\,{\buildrel \Delta \over =}\,}
\numberwithin{equation}{section}
\def\blue#1{\textcolor{blue}{#1}}
\def\red#1{\textcolor{red}{#1}}


\begin{document}

\include{ThesisFrontPage}
%%%%%%%%%%%%%%%%%%%%%%%%%%%%%%%%%%%%%%%%%%%%%%%%%%%%
%\text{}\newpage
%%%%%%%%%%%%%%%%%%%%%%%%%%%%%%%%%%%%%%%%%%%%%%%%%%%%


\newpage
%\newpage
%\include{ThesisNotations}
%%%%%%%%%%%%%%%%%%%%%%%%%%%%%%%%%%%%%%%%%%%%%%%%%%%%
\fancyhead{}
\fancyfoot{}
\pagestyle{fancy} 
%\fancyhead{\sffamily\small \thepage}
%\fancyhead{\sffamily\small \nouppercase{\rightmark}}
\fancyhead[RO,LE]{\sffamily\small \thepage}
\fancyhead[LO,RE]{\sffamily\small \nouppercase{\rightmark}}
\renewcommand{\headrulewidth}{0.4pt}
\renewcommand{\footrulewidth}{0.0pt}
%%%%%%%%%%%%%%%%%%%%%%%%%%%%%%%%%%%%%%%%%%%%%%%%%%%%
%\include{Thesis_Part1_Introduction}
%\include{Thesis_Part2_Tools}
%\include{Thesis_Part3_Large}
%\include{Thesis_Part4_Small}




%%%%%%%%%%%%%%%%%%%%%%%%%%%%%%%%%%%%%%%%%%%%%%%%%%%%
\begin{abstract}
Gut microbiota, composed by bacteria, fungi and viruses, play an essential role in the health of human and animals. Gut microbiota in infant can take effort to the gut development in adult. There is a dynamic process between bacteria and bacteriophages in early life, however, the interaction between the two important composition is unclear. In this project, we will use computational methods to study if CRISPR and Anti-CRISPR take part in the process and what the role it takes based on data from public database.

 Keywords: Infant gut microbiota CRISPR Anti-CRISPR
\end{abstract}
%%%%%%%%%%%%%%%%%%%%%%%%%%%%%%%%%%%%%%%%%%%%%%%%%%%%
\section{Introduction}
The mammalian microbiome consists of bacteria, archaea, fungi and viruses\citep{lynch2016human}. They have a profound influence on human physiology, nutrition and disease\citep{qin2010human}. They form a bioreactor and produce bioactive compounds. These compounds signal to distant organs including brain, liver, heart,lung and so forth\citep{schroeder2016signals}. In heart, gut micro-derived metabolites are recognized as contributors to atherogenesis\citep{wang2011gut}.  Microbiomes play a role in neurodevelopmental and mood disorders\citep{de2014food}. Differences in species richness and their diversity between autism spectrum disorder (ASD) and controls have been reported\citep{williams2012application}. As one part of the gut microbiome, the virome has a significant implication in health and disease\citep{virgin2014virome}. during inflammatory bowel diseases(IBD) the intestinal phage population is altered and transitions from an ordered state to a stochastic dysbiosis in murine\citep{duerkop2018murine}.

The microbiota of infants is a key factor for the development of the microbiome\citep{stewart2018temporal}. Epidemiological studies have shown that factors that alter bacterial communities in infants during childhood increase the risk for several diseases\citep{tamburini2016microbiome}. Childhood overweight is associated with the microbiota of early life in Denmark\citep{ajslev2011childhood} and Norway\citep{stanislawski2018gut}. Early gut microbiota was associated with infant growth rates.\citep{white2013novel} As for immune system, a diverse early life intestinal microbiota can inhibit the pathways that lead to allergic sensitization by multiple mechanisms\citep{reynolds2017early}. Besides, researches over the past few years reveal that the gut microbiome plays a role in basic neurogenerative processes and modulates many aspects of animal behavior in early life\citep{sharon2016central}.

The ecology of the bacterial microbiome increases in richness and diversity towards an adult-like composition\citep{yatsunenko2012human}. Soon after birth, the bacterial microbiome rapidly switches from predominantly facultative anaerobic bacteria to a diverse community of anaerobes\citep{koenig2011succession}. During the first months of life, the early infant gut bacteriophage virome is composed of a rich community of bacteriophages, the majority of which derive from the Caudovirales order. Subsequently, the bacteriophages decrease in richness and shifts towards a Microviridae-dominated community over the first 2 years of life. Thus, the infant virome and bacterial microbiome evolves in a dynamic trajectory during the early years of life\citep{lim2015early}.

As the Red Queen hypothesis proposes that organisms must continually evolve new mechanisms of resistance to parasites to avoid extinction\citep{liow2011red}. Bacteria have evolved a great of diverse strategies to defend themselves against phage, including restriction–modification enzymes that inactivate target DNA by cleavage, toxin–antitoxin modules that lead to phage abortive infection and CRISPR–Cas systems that target and inactivate specific nucleic acid sequences by cleavage. In response, phages have evolved various mechanisms to overcome these defenses, including expression of proteins that modify restriction sites\citep{kruger1983bacteriophage} or degrade restriction– modification cofactors\citep{studier1976samase}, antitoxin molecules that inhibit the activity of toxin‑antitoxin abortive infection systems\citep{otsuka2012dmd} and proteins that directly bind to and inactivate CRISPR–Cas machinery\citep{pawluk2016naturally}.

Thus, CRISPR-Cas and anti-CRISPR of the metagenomic and 16S rRNA sequences from eight healthy infants (four twin pairs) between 0-2 years old \citep{lim2015early} will be studied. Computational approaches will be used to detect CRISPR-Cas and anti-CRISPR sequences. What kind of CRISPR-Cas and anti-CRISPR are contained and during the dynamic process of bacteria-bacteriophage, if the CRISPR and Anti-CRISPR systems change, if they play a role will be answered. If the switch of lysogenic and lytic life style drives the dynamic process will be studied. Besides, in terms of previous researches, there is no CRISPR system in Bacteroidetes and no anti-CRISPR in Caudovirales62 which are dominant in gut, new insight is wished. 

Extrally, Maybe we can study the known interactions between anti-CRISPR proteins and their binding site of CRISPR-Cas, and then predict the novel anti-CRISPR proteins based on the numerous CRISPR-Cas data\citep{alipanahi2015predicting}.

%%%%%%%%%%%%%%%%%%%%%%%%%%%%%%%%%%%%%%%%%%%%%%%%%%%%
\section{Materials \& Methods}
Customized metagenomic and 16S rRNA sequences from eight healthy infants (four twin pairs) between 0-2 years old\citep{lim2015early} can be downloaded. Some operations have been done previous researchers. And the methods of virome sequencing processing, virome analysis and 16S rRNA gene analysis have been described\citep{lim2015early}. And the Methods of CRISPR analysis including indentification CRISPR spacers and indentification of phage contifs have been described\citep{stern2012crispr}. As for Anti-CRISPR analysis, There are two strategies for detecting Anti-CRISPR. First, using the Anti-CRISPR database to query the metagenomic data using BLASTN with e-value threshold of 1E-4. Second, using the aca gene sequences combined from all the previous researches to query the metagenomic data using the same parameters. There are two principles\citep{stern2012crispr} that can lead to the Python script for detection of the lysogenic phages. First, the lysogenic phages should be combined with integrase and recombinase. Second, the lysogenic phages should be integrated into bacteria with flanking sequences. And for the left phages, we regard them as lytic phages.


\section{Budget}
Probably, this project don't need budget.
    
Supervisor Tim G. Barraclough, Stineke Van Houte and Edze Westra has seen and accepted budget.

Signature:

%%%%%%%%%%%%%%%%%%%%%%%%%%%%%%%%%%%%%%%%%%%%%%%%%%%%
%%%%%%%%%%%%%%%%%%%%%%%%%%%%%%%%%%%%%%%%%%%%%%%%%%%%
\newpage
\bibliographystyle{plainnat}
\bibliography{proposal}

\end{document}